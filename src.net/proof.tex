\documentclass[10pt]{article}
\usepackage[T1]{fontenc}
\usepackage{lmodern}
\usepackage{url}
\usepackage{verbatim}
\usepackage{listings}
\usepackage{courier}
\usepackage[usenames,dvipsnames]{color}
\usepackage{textcomp}
\usepackage{hyperref}
\usepackage{xspace}
\usepackage{synttree}
\usepackage{amsthm}
\usepackage{fourier-orns}

\lstset{language={[Sharp]C}}
\lstset{upquote=true}
\lstset{basicstyle=\footnotesize\ttfamily,keywordstyle=\color{blue},extendedchars=true}
\newcommand{\inl}{\lstinline[breaklines=true]}
\newtheorem*{invariant}{Invariant}

\begin{document}

\title{Proof of Jump Algorithms Correctness}
\author{Michael Welch}
\maketitle

\section*{Introduction}
In this Refactoring study group the emphasis is on making existing code
better and having tests in place to prove that a refactoring did not
break the code.

However, when writing new code, one should not rely solely on testing
to prove their code is correct (no matter how much TDD proponents may say
so). One should think about their code ahead of time. In particular,
for any algorithm one writes, one can sketch out
the code and write a proof that the algorithm is correct.

In my implementation of Brainmess, there were only two methods that
were not completely trivial, and therefore I thought about the proofs.
The two methods are JumpForward and JumpBackward on my ProgramStream
class. We will not write proofs for our code. I'm not that academic.
However, you should be going thru a thought process like this as you decide
if your algorithms are correct. It actually takes less time to think 
thru all of this, then it does to type it all up and explain it.

\section*{Loop Invariant Proofs}
The proof of my JumpForward method, employs a technique called the
proof by loop invariant. This technique requires that one
identify some important property that the loop has, show that this
property is true initially and after every iteration of the loop,
and finally show that the loop terminates. It is very
much like an inductive proof.

The proof must show three things:

\begin{enumerate}
\item Initialization: The property holds before executing the loop, after initilization is complete.

\item Maintenance: The property holds after each iteration of the loop.

\item Termination: The loop eventually terminates.
\end{enumerate}

Let's take a look at my algorithm for JumpForward. This algorithm
is written as a function. (My actual implementation is a method
that accesses fields. This implementation stands alone and is easier
to talk about.) The function takes a program as a string, and
the current program counter as an int. It returns the new value to use
for the program counter after the jump is executed.

The listing for the algorithm is in figure~\ref{fig:jumpforward}.

\begin{figure}
\caption{The listing of JumpForward}
\label{fig:jumpforward}
\begin{lstlisting}[numbers=left]
internal int JumpForward(string prog, int pc)
{
    int nestLevel = 1;

    while (nestLevel > 0)
    {
        var instruction = prog[pc];
        
        if (instruction == ']') nestLevel--;
        else if (instruction == '[') nestLevel++;
        
        pc++;
    }
	
    return pc;
}
\end{lstlisting}
\end{figure}

\begin{proof}
The following sections will lay out the proof.

\subsection*{Preconditions and Postconditions}
First, we need to identify the preconditions and postconditions. The
precondition is that there is a \inl!'['! character at index value $q$, and
that \inl!pc! currently has a value of $q + 1$. The
character at index $q$ represents the location we are jumping from.
(I used a \inl!Debug.Assert! in my actual code to make sure this condition
always holds.) 

Another precondition is that \inl!program! is a well-formed program with
matching jump bracket characters. While we can (and will) say something
about what will happen in the presence of non-matching brackets, we
don't really care if the algorithm works or not in that case.

Given the second precondition we know that the character at index value
$q$ has a matching \inl!']'! character at some index value. We'll name
this location $r$ even though we don't know its specific value (that
is what we will determine).

The postcondition is that when we are done \inl!pc! will
contain the value $r + 1$ (or in the case of a malformed program
the program eventually terminates with an index out of bounds exception).
This is what we will prove.

\subsection*{Loop Invariant Property}
The property of the loop that is important to our proof is the following: 

\begin{invariant}
The \inl!nestLevel! always correctly indicates the nesting level of the
program counter relative to the character at position $q$. 
\end{invariant}

A value of 0 means
we are outside of the loop. A value of 1 means we are inside the loop.
A value of 2 means we are inside of a loop that is nested inside
of the loop.

This is the invariant property. We will show that it holds before entering the
loop.  We will also show that if it holds at the beginning of an
iteration then it will still hold after each iteration of the loop.


\subsection*{Initialization}
The invariant is trivially true after initialization. By the precondition,
we know that the program counter is at $q + 1$ and therefore we are at a
nesting level of 1 and thefore we initialize \inl!nestLevel! to the value 1
on line 3.

\subsection*{Maintenance}
We assume that \inl!nestLevel! accurately
reflects our current nesting level at beginning of an iteration.

We test the condition that \inl!nestLevel > 0! is true. Inside of the loop
we get the instruction at the program counter.

We then have 3 cases to consider. The instruction is a \inl!'['!, or a
\inl!']'! or some other character. No matter what we increment \inl!pc!  on
line 12.

So the 3 cases are:

\begin{enumerate}
\item If the instruction is a \inl!'['! then we are entering a nested loop and
\inl!nestLevel! is incremented to reflect that. See line 9. 

\item If the instruction is a \inl!`]'! then we are leaving a nested loop and
\inl!nestLevel! is decremented to reflect that. See line 10.

\item If we see any other instruction the nesting doesn't change and
the \inl!nestLevel! is not modified.
\end{enumerate}

So we see that no matter what happens in the loop, the \inl!nestLevel!
accurately represents the nesting level when we end the iteration.

\begin{center}
\framebox[4in]{
\begin{minipage}[t]{3.5in}
{\Large{\bomb}}
Note, there is a fourth case which only happens if a program is
malformed. It is possible that the program counter has been incremented
off the end of the program and we get an exception when retrieving the 
insruction. This is guaranteed (by definition and this proof) not to happen 
if the program is well-formed. 
\end{minipage}
}
\end{center}

\subsection*{Termination}
In general a bounding function will give us an upper bound on how many 
iterations are left to execute and allows us to show that our algorithm isn't
an infinite loop. After each iteration of the loop the value of this 
function must get smaller. If it every stays the same (or gets bigger) then
there is a chance that we have an infinitie loop.

Recall that based on our preconditions there is a \inl!'['! at index $q$,
that \inl!pc! has a value of $q + 1$ and that we know there is a matching
\inl!']'! at some index value $r$. On each iteration, we get one step closer
to the matching bracket. So our bounding function is $r - pc + 1$.

This bounding function tells us that we will execute the loop a finite
number of times. And it tells us that it gets smaller with each iteration.
This can be seen by remembering that the program counter is incremented on
every iteration.

Let's check this on the trivial case. Assume we have the program
\inl!"[]"!. In this case $q=0$, $pc=1$, and $r=1$. Then at the beginning
of the loop the value of our bounding function is $r - pc + 1$ or
$1 - 1 + 1 = 1$. This is correct. We will run the loop one time before
finding our match.


\subsection*{Putting it all Together}
We have shown that our loop has an important property. It maintains the value
of \inl!nestLevel! so that it always correctly indicates our nesting level
relative to our start character. 

We also know by inspection that we exit the loop as soon as the nesting
level hits 0. And this only happens after we have processed the
matching \inl!']'! character at position $r$. We also know that 
\inl!pc! will have the value of $r+1$ at this time. (To see this fact
let's consider our last iteration. We read the \inl!']'! character at position
$r$ on line 7. On line 9 we decrement the nest level to 0. Then on line 12 we
update the program counter to $r+1$.)

This is the postcondition we meant to demonstrate. So we have shown that this
algorithm successfully locates and returns the instruction after the matching
bracket character.
\end{proof}
\end{document}
